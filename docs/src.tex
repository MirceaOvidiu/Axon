\documentclass[12pt, a4paper]{report}

% --- Preamble: Packages for Romanian support and formatting ---
\usepackage[utf8]{inputenc}
\usepackage[romanian]{babel} % Sets titles like "Contents" to "Cuprins"
\usepackage[margin=1in]{geometry}
\usepackage{titlesec}

% Optional: Set line spacing to 1.5 (common for academic papers)
\usepackage{setspace}
\onehalfspacing

\begin{document}

% --- Automatically generates the Table of Contents ---
\tableofcontents
\newpage

% --- Chapter 1 ---
\chapter{Introducere}
\section{Contextul general și motivația cercetării}
\hspace{0.75cm}Evoluția omului a constat în dezvoltarea gândirii critice, structurate, logice cu scopul de a depăși 
impedimentele puse în calea existentei acestuia. 
Cu cât sfera cunoasterii s-a largit, indivizii au fost nevoiți să se specializeze în anumite domenii,
cum ar fi agricultura, astronomia, arhitectura sau medicina. 
Pentru a putea continua evolutia in aceste domenii, apare necesitatea individului cu o 
gândire pretata pentru probleme complexe interdisciplinare: anume, inginerul. 
Urmărind avansul cunoștințelor din domeniul ingineriei, această 
meserie s-a ramificat în diferite directii, precum: mecanica, energetica, chimica, electronică, telecomunicații, 
automatică, etc.

\hspace{0.75cm}Automatica reprezintă o știință interdisciplinară care are o viziune holistică asupra sistemelor complexe. 
Scopul acesteia este de a optimiza procesele si evolutia acestor sisteme, minimizand contribuția si eroarea umana.
Unul dintre cele mai complexe sisteme cunoscute de umanitate este reprezentat de insusi corpul omenesc. 

\hspace{0.75cm}O perturbatie in urma careia acest sistem rareori revine la un regim de funcționare normal  
este accidentul vascular cerebral. La nivel global, AVC-ul reprezintă o problemă majoră de sănătate publică, 
fiind a doua cauză de mortalitate și a treia cauză principală de dizabilitate pe termen lung, conform studiului 
Global Burden of Disease \cite{gbd2021}. Efectele unui AVC sunt profunde și adesea ireversibile, 
privând rapid porțiuni ale creierului de oxigen și nutrienți,
ceea ce duce la necroză tisulară și la pierderea unor funcții motorii, cognitive sau senzoriale esențiale.

\hspace{0.75cm}În România, situatia este critică, incidența și gravitatea acestui fenomen atingând cote semnificativ
mai mari  în comparație cu media europeană. Statisticile plasează România în topul țărilor europene în ceea ce 
privește mortalitatea cauzată de afecțiuni cerebrovasculare, rata de deces prin AVC fiind de aproximativ două ori
mai mare decât media Uniunii Europene și reprezentând peste 20\% din totalul deceselor la nivel național 
\cite{romania_stats}. Anual, aproximativ 60 de mii de pacienți români suferă un accident vascular cerebral ischemic sau hemoragic. 

\hspace{0.75cm}Un alt capitol la care Romania se afla in coada clasamentelor europene este reprezentat de accesul facil la 
infrastructura medicală și tratamente de recuperare. Majoritatea centrelor de recuperare neurologică bine echipate 
și a specialiștilor cu experiență în neuroreabilitare sunt concentrate în marile centre universitare \cite{tiu_2023}.

\section{Obiectivele proiectului}

\hspace{0.75cm} Luând în considerare problemele menționate anterior, în cadrul acestei lucrări este abordată 
următoarea soluție: realizarea unui sistem telemedical integrat, compus dintr-o aplicație de tip wearable (pentru smartwatch) 
și o aplicație mobilă, care să monitorizeze, să cuantifice și să personalizeze de la distanță procesul de recuperare motorie 
a pacienților, procesul de reabilitare fiind asistat de algoritmi de procesare de semnal și modele de tip machine learning.

\hspace{0.75cm}Obiectivele specifice ale proiectului sunt: 
\begin{itemize}
    \item Dezvoltarea unei aplicații pentru platforma Wear OS, capabilă să monitorizeze în 
     timp real mișcările și semnalele fiziologice ale pacienților.
    \item Integrarea algoritmilor specifici ingineriei biomedicale în vederea evaluării și cuantificării 
     numerice obiective a calității mișcărilor efectuate de pacient. 
    \item Integrarea algoritmilor de machine learning astfel încât sistemul să poată calcula un progres realist 
     și să ajusteze automat obiectivele de recuperare ale pacientului pe diferite orizonturi de timp
     (zilnic, săptămânal, lunar).
    \item Oferirea unui mod de vizualizare intuitiv care să permită o metodă ușoară de urmărire a 
     progresului și de comparare a sesiunilor de recuperare, atât pentru pacient, cât și pentru personalul medical. 
    \item Asigurarea securității și confidențialității datelor medicale colectate  
\end{itemize}

% --- Chapter 2 ---
\chapter{Patologia si recuperarea post-AVC}

\section{Patologia accidentului vascular cerebral}
\hspace{0.75cm}Accidentul vascular cerebral reprezintă un deficit neurologic focal cu instalare bruscă, cauzat de o perturbare severă 
a circulației sanguine la nivelul encefalului. Din punct de vedere fiziopatologic, AVC-ul se clasifică în două categorii 
majore: ischemic (care reprezintă aproximativ 85\% din cazuri, cauzat de ocluzia unui vas de sânge prin tromboză sau embolism)
și hemoragic (cauzat de ruptura vasculară) \cite{stroke_patho_2020}. Indiferent de etiologie, privarea țesutului cerebral de
oxigen și glucoză declanșează o cascadă ischemică distructivă. În centrul leziunii neuronii 
necrozeaza în câteva minute. Supraviețuirea pacientului este adesea acompaniată
de deficite motorii severe pe partea opusă emisferei cerebrale afectate (hemipareză sau hemiplegie), tulburări de echilibru 
și spasticitate musculară, care compromit drastic independența funcțională a acestuia.

\section{Mecanismul biologic al recuperării: neuroplasticitatea}

\hspace{0.75cm}Spre deosebire de dogma medicală clasică, care susținea că sistemul nervos central adult este o structură rigidă, cercetările
moderne au demonstrat că recuperarea funcțiilor motorii pierdute este posibilă datorită neuroplasticității. Aceasta este definită ca 
abilitatea intrinsecă a creierului de a se remodela structural și funcțional ca răspuns la experiență, învățare sau leziuni \cite{kleim_jones_2008}.
După un AVC, creierul inițiază un proces de reorganizare corticală: neuronii din zonele adiacente leziunii, sau chiar din emisfera 
contralaterală (sănătoasă), pot prelua parțial funcțiile rețelelor neuronale distruse. Acest fenomen implică formarea de noi sinapse 
(sinaptogeneză), modificarea eficienței sinapselor existente și ramificarea dendritică.

\hspace{0.75cm}La nivel național și internațional, contribuții fundamentale la aprofundarea acestor mecanisme
celulare și a capacității creierului de autoreparare 
au fost aduse de academicianul și neurochirurgul român Prof. Dr. Leon Dănăilă \cite{danaila_2023}.
Desi structura cerebrală dispune de acest instrumentar celular uimitor de regenerare, neuroplasticitatea rămâne un proces
„dependent de experiență”; ea nu se produce spontan
la un nivel suficient de înalt pentru a reda independența funcțională, ci trebuie stimulată activ
și susținut prin reînvățare motorie \cite{cramer_2011}.

\section{Importanța mișcărilor repetitive}
\hspace{0.75cm}Pentru a transforma potențialul neuroplasticității în recuperare funcțională reală, reabilitarea motorie trebuie să respecte 
anumite principii biomecanice și neurologice stricte. Unul dintre cele mai importante este principiul repetiției. Studiile de neurofiziologie 
arată că sunt necesare mii de repetări ale unei mișcări specifice pentru a consolida noile căi neuronale și a transforma o mișcare stângace, 
conștientă, într-un automatism motor \cite{lang_2009}. Terapia indusă prin constrângere (CIMT - Constraint-Induced Movement Therapy) 
și antrenamentul specific sarcinii (task-specific training) demonstrează că intensitatea și frecvența ridicată a exercițiilor dictează 
gradul de recuperare. În lipsa unor mișcări repetitive, ghidate corect, pacientul riscă să dezvolte mișcări compensatorii vicioase 
(folosind doar membrele sănătoase), 
fenomen cunoscut sub numele de „neutilizare învățată” (learned non-use), care plafonează recuperarea pe termen lung \cite{taub_2014}.

% --- Chapter 3 ---
\chapter{Stadiul actual al tehnologiei de recuperare}
\section{Dispozitive wearable si sensori in monitorizarea sanatatii}
\section{Sisteme existente de reabilitare - analiza state of the art}
\section{Analiza comparativa a solutiilor vs. solutia propusa}

\chapter{Fundamentele teoretice ale analizei miscarii}
\section{Senzori inertiali (IMU): accelerometru si giroscop}
\section{Metrici de evaluare a calitatii miscarii}
\section{Analiza fluiditatii: Algoritmul Spectral Arc Length - SPARC}
\section{Analiza stabilitatii: Log Dimensionless Jerk - Acceleration - LDLJ - A}
\section{Recunoasterea tiparelor si corectitudinii: Dynamic Time Warping - DTW}
\section{Variabilitatea ritmului cardiac (HRV) si activitate electrodermala (EDA)}

% --- Chapter 5 (Solution) ---
\chapter{Soluția propusă}
\section{Prezentarea generală a ecosistemului}
\section{Componenta Hardware: Specificații Google Pixel Watch 3 și limitări}
\section{Componenta Software: Arhitectura aplicației Wear OS și Mobile}
\section{Arhitectura Cloud și fluxul de date - GCP}
\section{Interpretarea rezultatelor}
\section{Securitatea și confidențialitatea datelor medicale}


% --- Chapter 6 ---
\chapter{Implementarea solutiei}
\section{Tehnologii și medii de dezvoltare utilizate}
\section{Achiziția și procesarea datelor senzorilor}
\section{Implementarea algoritmilor de evaluare}
\section{Calculul scorului de fluiditate - SPARC}
\section{Mecanismul de feedback haptic și vizual în timp real}
\section{Dezvoltarea Dashboard-ului pentru monitorizarea progresului}

% --- Chapter 7 ---
\chapter{Testare, Rezultate, Studiu de caz}
\section{Metodologia de testare și scenariile utilizate}
\section{Organizarea seturilor de date experimentale}
\section{Analiza performanței tehnice}
\section{Interpretarea rezultatelor }
\section{Studiu de caz: Evoluția unui set de exerciții}

% --- Chapter 8 ---
\chapter{Concluzii}
\section{Limitări ale sistemului actual}
\section{Direcții de dezvoltare ulterioară}

% --- Bibliography ---
\begin{thebibliography}{99}

\bibitem{gbd2021} 
V. L. Feigin et al., \textit{Global, regional, and national burden of stroke and its risk factors, 1990–2021: a systematic analysis for the Global Burden of Disease Study 2021}, The Lancet Neurology, 2024.

\bibitem{romania_stats} 
A. M. Ionescu et al., \textit{A statistical analysis of acute ischemic stroke before and during the COVID-19 pandemic}, Romanian Journal of Medical Practice, 2022.

\bibitem{eu_stroke_2022} 
S. Wassertheil-Smoller et al., \textit{Burden of Stroke in Europe: An Analysis of the Global Burden of Disease Study Findings From 2010 to 2019}, American Heart Association Journals, 2022.

\bibitem{tiu_2023} 
C. Tiu et al., \textit{Quality of acute stroke care in Romania: Achievements and gaps between 2017 and 2022}, Frontiers in Neurology, Vol. 13, 2023.

\bibitem{ms_protocol_2025} 
Ministerul Sănătății din România, \textit{Referat de aprobare: Protocol național de practică medicală privind tratamentul intervențional al pacienților cu accident vascular cerebral acut}, București, Februarie 2025.

\bibitem{stroke_patho_2020}
A. D. Mendelson and S. Prabhakaran, \textit{Diagnosis and Management of Transient Ischemic Attack and Acute Ischemic Stroke: A Review}, JAMA, vol. 325, no. 11, pp. 1088-1098, 2021.

\bibitem{penumbra_2018}
J. Astrup, B. K. Siesjö, and L. Symon, \textit{Thresholds in cerebral ischemia - the ischemic penumbra}, Stroke, vol. 12, no. 6, pp. 723-725, 1981. (Notă: Lucrarea clasică care a definit penumbra).

\bibitem{kleim_jones_2008}
J. A. Kleim and T. A. Jones, \textit{Principles of experience-dependent neural plasticity: implications for rehabilitation after brain damage}, Journal of Speech, Language, and Hearing Research, vol. 51, no. 1, pp. S225-S239, 2008.

\bibitem{cramer_2011}
S. C. Cramer et al., \textit{Harnessing neuroplasticity for clinical applications}, Brain, vol. 134, no. 6, pp. 1591-1609, 2011.

\bibitem{lang_2009}
C. E. Lang, J. R. MacDonald, D. S. Reisman, L. Boyd, T. Jacobson Kimberley, S. M. Schindler-Ivens, A. W. Hornby, J. C. Ross, and P. L. Scheets, \textit{Observation of amounts of movement practice provided during stroke rehabilitation}, Archives of Physical Medicine and Rehabilitation, vol. 90, no. 10, pp. 1692-1698, 2009.

\bibitem{taub_2014}
E. Taub, G. Uswatte, and V. W. Mark, \textit{The functional significance of cortical reorganization and the parallel development of CI therapy}, Frontiers in Human Neuroscience, vol. 8, p. 396, 2014.

\bibitem{kwakkel_2017}
G. Kwakkel et al., \textit{Standardized measurement of sensorimotor recovery in stroke trials: Consensus-based core recommendations from the Stroke Recovery and Rehabilitation Roundtable}, International Journal of Stroke, vol. 12, no. 5, pp. 451-461, 2017.

\bibitem{peters_2021}
D. Ding, J. Cooper, C. M. Pasquina, and L. Finkelstein, \textit{Sensor technology for stroke rehabilitation: towards tele-rehabilitation and home-based therapy}, IEEE Reviews in Biomedical Engineering, vol. 14, pp. 163-176, 2021.

\bibitem{danaila_2023}
L. Dănăilă, \textit{Neuroplasticitatea}, Editura Academiei Române, București, 2023.

\bibitem{pais_danaila_2013}
V. Păiș, L. Dănăilă, and E. Păiș, \textit{Cordocytes-stem cells cooperation in the human brain with emphasis on pivotal role of cordocytes in perivascular areas of broken and thrombosed vessels}, Ultrastructural Pathology, vol. 37, no. 6, pp. 425-432, 2013.

\end{thebibliography}

\end{document}