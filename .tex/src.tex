\documentclass[12pt, a4paper]{report}

% --- Preamble: Packages for Romanian support and formatting ---
\usepackage[utf8]{inputenc}
\usepackage[romanian]{babel} % Sets titles like "Contents" to "Cuprins"
\usepackage[margin=1in]{geometry}
\usepackage{titlesec}

% Optional: Set line spacing to 1.5 (common for academic papers)
\usepackage{setspace}
\onehalfspacing

\begin{document}

% --- Automatically generates the Table of Contents ---
\tableofcontents
\newpage

% --- Chapter 1 ---
\chapter{Introducere}
\section{Contextul general și motivația cercetării}
\section{Obiectivele proiectului}
\section{Structura tezei}

% --- Chapter 2 ---
\chapter{Recuperarea post-AVC}
\section{Patologia accidentului vascular cerebral}
\section{Mecanismul biologic al recuperării: neuroplasticitatea}
\section{Importanța mișcărilor repetitive}
\section{Limitări ale metodelor tradiționale de recuperare}

% --- Chapter 3 ---
\chapter{Stadiul actual al tehnologiei de recuperare}
\section{Dispozitive wearable si sensori in monitorizarea sanatatii}
\section{Sisteme existente de reabilitare - state of the art}
\section{Analiza comparativa a solutiilor vs. solutia proupusa}

% --- Chapter 4 (Note: You skipped 4 in your list, LaTeX will auto-number this as 4) ---
\chapter{Fundamentele teoretice ale analizei miscarii}
\section{Senzori inertiali (IMU): accelerometru si giroscop}
\section{Metrici de evaluare a calitatii miscarii}
\section{Analiza fluiditatii: Algoritmul Spectral Arc Length - SPARC}
\section{Analiza stabilitatii: Log Dimensionless Jerk - LDLJ}
\section{Recunoasterea tiparelor si corectitudinii: Dynamic Time Warping - DTW}
\section{Variabilitatea ritmului cardiac (HRV) si activitate electrodermala (EDA)}

% --- Chapter 5 (Solution) ---
\chapter{Soluția propusă}
\section{Prezentarea generală a ecosistemului (Diagramă High-Level)}
\section{Componenta Hardware: Specificații Google Pixel Watch și limitări}
\section{Componenta Software: Arhitectura aplicației Wear OS și Mobile}
\section{Arhitectura Cloud și fluxul de date (Telemetry Pipeline)}
\section{Interpretarea rezultatelor}
\section{Securitatea și confidențialitatea datelor medicale}


% --- Chapter 6 ---
\chapter{Implementarea solutiei}
\section{Tehnologii și medii de dezvoltare utilizate (Tech Stack)}
\section{Achiziția și pre-procesarea datelor senzorilor (Sensor Fusion) 6.2.1. Calibrarea și filtrarea semnalelor (50Hz - 200Hz)}
\section{Sincronizarea datelor IMU cu cele biometrice}
\section{Implementarea algoritmilor de evaluare}
\section{Calculul scorului de fluiditate (Implementare SPARC)}
\section{Detectarea erorilor de execuție}
\section{Mecanismul de feedback haptic și vizual în timp real}
\section{Dezvoltarea Dashboard-ului pentru monitorizarea progresului}

% --- Chapter 7 ---
\chapter{Testare, Rezultate, Studiu de caz}
\section{Metodologia de testare și scenariile utilizate}
\section{Organizarea seturilor de date experimentale}
\section{Analiza performanței tehnice (Latență, Consum baterie)}
\section{Interpretarea rezultatelor (Corelația scorului aplicației cu execuția corectă)}
\section{Studiu de caz: Evoluția unui set de exerciții}

% --- Chapter 8 ---
\chapter{Concluzii}
\section{Sinteza contribuțiilor personale}
\section{Limitări ale sistemului actual}
\section{Direcții de dezvoltare ulterioară}

% --- Bibliography ---
\begin{thebibliography}{99}
\bibitem{example1} 
Author Name, \textit{Book Title}, Publisher, Year.
\end{thebibliography}

\end{document}